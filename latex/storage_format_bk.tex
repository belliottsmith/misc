\documentclass[fleqn]{article}

\usepackage{hyperref}
\usepackage[utf8]{inputenc}
\usepackage{listings}
\usepackage{color}
\usepackage{xcolor}
\usepackage{amssymb}
\usepackage{amsmath}
\usepackage{setspace}
\usepackage{multirow}
\usepackage{algorithm}
\usepackage{xpatch}
\usepackage[noend]{algpseudocode}

\makeatletter
\def\BState{\State\hskip-\ALG@thistlm}
\xpatchcmd{\algorithmic}{\itemsep\z@}{\itemsep=0.5ex plus1pt}{}{}
\algnewcommand{\LineComment}[1]{\Statex \hskip\ALG@thistlm #1}
\makeatother

\title{Cassandra Storage Format\\Proposal for 3.1\\Version 1}
\author{Benedict Elliott Smith}
\date{January 2015}

\begin{document}

\maketitle

\begin{abstract}
This document is a draft proposal for a generalised replacement for the sstable format used in Cassandra.
The goal is to deliver a framework building on top of a number of abstracted components, with which
any efficient sstable representation can be encapsulated. It is designed to directly align with modern CQL
data modelling, however it should also support thrift models at least as efficiently.
\end{abstract}

\small

\section{Outline}
This section will outline the components and some associated concepts that are common to
\\
\paragraph{Definitions}
\subparagraph{$PartitionEpoch$, $RowEpoch$}
\subparagraph{}
    Against each tier of our structure we will encode the minimum timestamp found in the next
    tier, so that each may be encoded more efficiently. The expectation is that this will permit 
    many cells to store either only a few bytes, a single bit, or no data at all, to represent their timestamp.
\subparagraph{$ColId$}
\subparagraph{}
    A translation from column to a unique integer id

\paragraph{Components}
\paragraph{}
These components will each have more than one implementation.
\subparagraph{$PartitionIndex$}
\subparagraph{}
    One per sstable.
    Performs a translation from $PartitionKey$ to a set of addresses in $DataFile$ that may contain
    data for the key. This translation is permitted to yield false positives, but should
    expect to produce either zero or one result on average, with no false negatives.
    Multiple keys may map to the same location in $DataFile$, which can be exploited by an implementation
    to produce a compact mapping. These keys will generally be adjacent, but some may be out-of-order
    due to page packing. Its persistence is managed separately.
    \\
    \begin{algorithmic}[2]
    \scriptsize
    \Function{Lookup}{$PartitionIndex, Token$}
    \Statex \Return All positions possibly containing $PartitionKey$
    \EndFunction
    \Statex{\bf function} \textsc{Range} ({$PartitionIndex, StartToken, EndToken$})
    \Statex \Return ${Page \mid \forall (Token \mapsto Page) \in PartitionIndex, StartToken <= Key < EndToken }$
    \end{algorithmic}

\subparagraph{$RowIndex$}
\subparagraph{}
    One per partition. Accessed sequentially, located by a $PartitionIndex$ lookup, it performs 
    a translation from $RowKey$ to a $RowEpoch$ and a position to provide to each $ValueStore_N$
    to produce cell values. It is stored in $DataFile$.
    \\
    \begin{algorithmic}[2]
    \scriptsize
    \Function{Read}{$RowIndex, RowKey$}
    \Statex \Return $\langle$Offsets in each $DataFile$ at which $ValueStore$ persisted $RowKey$, $RowEpoch \rangle$
    \EndFunction
    \end{algorithmic}

\subparagraph{$ValueStore$}
\subparagraph{}
    One per partition. A sequential chunk of bytes that combines the position retrieved
    from the $RowIndex$ with an offset stored in $DataFile_0$, the queried columns, the $ColId$ translation
    and the $RowEpoch$ to return cell values. We assume a property $Pos$ can be queried prior to insertion
    of a new row that returns the value we will store in $RowIndex$ for future lookups. It is stored in $DataFile$.
    \\
    \begin{algorithmic}[2]
    \scriptsize
    \Function{Read}{$ValueStore, ColIds, Pos, RowEpoch$}
    \Statex \Return Values for the $ColIds$ persisted by $ValueStore$ at relative $Pos$
    \EndFunction
    \end{algorithmic}


\section{Implementation Details}
\subsection{Notation Key}
\paragraph{}
\begin{tabular}{l l}
$X \Leftarrow Y$ & Write/Add/Append $Y$ To $X$\\
$X \dashleftarrow Y$ & Read $X$ From $Y$\\
$X \gets Y$ & Set $X$ to $Y$\\
$\langle A,B \rangle$ & A tuple containing the values $A$ and $B$\\
$\{Item | Constraint\}$ & The set of all $Item$ produced by $Constraint$\\
$S_N$ & $\equiv \{S_n | n \in N\}$\\
$Stmt_x | \forall x \in X$ & Execute statement $S$ for each $x$ in $X$\\
$\lvert X \rvert$ & $\equiv \mathbf{card}(X)$, i.e. the number of elements in $X$\\
\end{tabular}

\subsection{Core Outline}
\small
This section will outline the core algorithm using the component definitions provided in the previous
section. A basic description of the algorithm follows.
\\\\
\paragraph{Description}

\begin{algorithm}
\scriptsize
\caption{Writing}
\begin{algorithmic}[1]
\Procedure{Write}{$Partitions$}
\State $PartitionIndex, DataFile \gets $ \Call{New}{}
\State $ColId \gets \{c \mapsto i \mid \forall c, d \in Columns, d < c \implies d \mapsto j \wedge j < i \}$
\State $Metadata \Leftarrow ColId$
\Statex
\For{$p \gets Partitions$} 
 \State $RowIndex, ValueStore \gets $ \Call{New}{}
 \State $PartitionEpoch \gets \min \{v.timestamp \mid v \in r.Values, r \in p.Rows\}$
 \Statex
 \For{$r \gets p.Rows$}
   \State $RowEpoch \gets \min \{v.timestamp \mid v \in r.Values\}$
   \State $EpochDelta \gets RowEpoch - PartitionEpoch$
   \State $RowIndex \Leftarrow \langle r.RowKey \mapsto \langle EpochDelta, ValueStore.Pos \rangle \rangle $ 
   \State $Values \gets \{ ColId(c) \mapsto r.Values(c) \mid c \in Columns \}$
   \State $v.Time \gets v.Time - RowEpoch \mid \forall v \in \mathbf{ran}(Values_{n})$
   \State $ ValueStore \Leftarrow Values $
 \EndFor
\Statex
 \State $Buffer \Leftarrow \langle p.PartitionKey, PartitionEpoch, RowIndex, ValueStore, DataFile.Pos \rangle$
 \State $Size \gets DataFile.pos - \min\limits_{\forall b \in Buffer} b.Pos $
 \If {$Size \geq PageSize$}
   \State \Call{Flush}{$Flush, PartitionIndex, DataFile_{N}$}
   \State $Buffer \gets $ \Call{New}{}
 \EndIf
\EndFor
\State \Call{Flush}{$Buffer, PartitionIndex, DataFile_{N}$}
\EndProcedure
\Statex
\Procedure{Flush}{$Buffer, PartitionIndex, DataFile_{N}$}
   \State $PartitionIndex \Leftarrow \langle $\Call{Token}{$PartitionKey$}$ \mapsto DataFile.Pos \rangle \mid \forall PartitionKey \in Buffer$
   \State $DataFile \Leftarrow \lvert Buffer \rvert$
   \State $DataFile \Leftarrow $ \Call{Token}{$PartitionKey$} $ \mid \forall \langle PartitionKey, \_,\_,\_\rangle \in Buffer \rangle$
   \State $Offset \gets 0$
   \For{$\langle PartitionKey, \_, \_, \_ \rangle \gets Buffer$}
     \State $DataFile \Leftarrow Offset$
     \State $Offset \gets Offset + $ \Call{SizeOf}{$\langle PartitionKey, RowIndex, ValueStore,$\textsc{Epoch}$,$\{\textsc{Address}\}$\rangle$}
   \EndFor
   \For{$\langle PartitionKey, PartitionEpoch, RowIndex, ValueStore \rangle \gets Buffer$}
    \State $DataFile \Leftarrow \langle PartitionEpoch, PartitionKey \rangle$
    \State $DataFile \Leftarrow DataFile.Pos + $ \Call{SizeOf}{$RowIndex$} $ + $ \Call{SizeOf}{\{\textsc{Address}\}}
    \State $DataFile \Leftarrow RowIndex$
    \State $DataFile \Leftarrow ValueStore$
   \EndFor
\EndProcedure
\end{algorithmic}
\end{algorithm}

\begin{algorithm}
\scriptsize
\caption{Reading}
\begin{algorithmic}[2]
\Function{Read}{$PartitionKey, Cols, RowKeys$}
\State $ColGroup, ColId \dashleftarrow Metadata$
\State $Token \gets $ \Call{Token}{$PartitionKey$}
\State $CandidatePages \gets $\Call{Lookup}{$PartitionIndex, Token$}
\For{$Page \gets CandidatePages$}
 \State $Count \dashleftarrow DataFile[Page]$
 \State $TokenBase \gets Page +  $ \Call{SizeOf}{\textsc{$Count$}}
 \State $OffsetsBase \gets TokenBase + Count \times $\Call{SizeOf}{\textsc{Token}}
 \For{$i \gets $ \Call{Find}{$DataFile[TokenBase \dots OffsetBase), Token$}}
  \State $OffsetPosition \gets OffsetsBase + i \times $ \Call{SizeOf}{\textsc{Offset}}
  \State $Offset \dashleftarrow DataFile[OffsetPosition]$
  \State $PartitionEpoch, CandidateKey, Pos_N \dashleftarrow DataFile[Offset]$
  \If{$PartitionKey = CandidateKey$} 
   \State $Offset \gets Offset + $ \Call{SizeOf}{$\langle PartitionEpoch, CandidateKey, Pos \rangle$}
   \State $RowIndex \gets DataFile[Offset]$
   \State \Return \Call{Build}{$RowKeys, PartitionEpoch, ReadGroup, ColId, RowIndex, Pos, DataFile$}
  \EndIf
 \EndFor
\EndFor
\Return \textbf{nil}
\EndFunction
\Statex
\Function{Build}{$RowKeys, PartitionEpoch, ColId, Pos, DataFile$}
\State $Partition \gets $ \Call{New}{}
\State $ValueStore \Leftarrow DataFile[Pos_n]$
\For{$RowKey \gets RowKeys$}
 \State $Pos, EpochDelta \gets$ \Call{Read}{$RowIndex, RowKey$}
 \State $RowEpoch \gets PartitionEpoch + EpochDelta$
 \State $Partition \Leftarrow $ \Call{Read}{$ValueStore, ColId(Columns), Pos, RowEpoch$}
\EndFor
\Return $Partition$
\EndFunction
\end{algorithmic}
\end{algorithm}

\clearpage
\subsection{Extension One - Column Groupings}
\small
This section will outline the components and some associated concepts, before they are all used to define 
a basic pseudocode implementation.
\\\\
\paragraph{Description}

\paragraph{Definitions}
\begin{itemize}
  \item (Column) $Group_N$\\[2pt]
    A subset of the columns defined on the table, that are stored together in a
    corresponding $DataFile_N$. By default there will be one group containing all columns, 
    but rarely accessed columns or columnar layouts may be separated into their own group.
    The same mechanism can be used to separate the row index from row values by using
    an empty initial group. It also supports duplication of cell data, for different access
    patterns.
  \item $ColId_N$\\[2pt]
    A translation from column to a unique integer id within each $Group_N$
    
\end{itemize}

\paragraph{Components}
\begin{itemize}
  \item $ValueStore_N$\\[2pt]
    One per $Group_N$, per partition. Otherwise as defined originally.
\end{itemize}

\begin{algorithm}
\scriptsize
\caption{Writing}
\begin{algorithmic}[1]
\Procedure{Write}{$Partitions, Group_N$}
\State $PartitionIndex, Buffer \gets $ \Call{New}{}
\State $DataFile_{n} \gets $ \Call{New}{} $\mid \forall n \in N$
\Statex
\State $ColGroup \gets \{c \mapsto n \mid \forall c \in Group_{n}, \forall n \in N \}$
\State $ColId_{n} \gets \{c \mapsto i \mid \forall c, d \in G, d < c \implies d \mapsto j \wedge j < i \} \mid \forall n \in N, G \equiv Group_{n}$
\State $Metadata \Leftarrow ColGroup$
\State $Metadata \Leftarrow ColId$
\Statex
\For{$p \gets Partitions$} 
 \State $RowIndex \gets $ \Call{New}{}
 \State $ValueStore_{n} \gets $ \Call{New}{} $ \mid \forall n \in N$
 \State $PartitionEpoch \gets \min \{v.timestamp \mid v \in r.Values, r \in p.Rows\}$
 \Statex
 \For{$r \gets p.Rows$}
   \State $RowEpoch \gets \min \{v.timestamp \mid v \in r.Values\}$
   \State $EpochDelta \gets RowEpoch - PartitionEpoch$
   \State $RowIndex \Leftarrow \langle r.RowKey \mapsto \langle EpochDelta, \{ v.Pos \mid v \in ValueStore_{N} \}\rangle \rangle $ 
   \State $Values_{n} \gets \{ ColId(c) \mapsto r.Values(c) \mid c \in Group_{n} \} \mid \forall n \in N$
   \State $v.Time \gets v.Time - RowEpoch \mid \forall v \in \mathbf{ran}(Values_{n}), \forall n \in N$
   \State $ ValueStore_{n} \Leftarrow Values_{N} $
 \EndFor
\Statex
 \State $Pos_{n} \gets DataFile_{n}.Pos | \forall n \in N$
 \State $Buffer \Leftarrow \langle p.PartitionKey, PartitionEpoch, RowIndex, ValueStore_{N}, Pos_{N} \rangle$
 \State $Size \gets \sum\limits_{n=0}^{N} (DataFile_{n}.pos - \min\limits_{\forall b \in Buffer} b.Pos_{n}) $
 \If {$Size \geq N \times PageSize$}
   \State $Flush \gets$ \Call{Select}{$Buffer$}
   \State \Call{Flush}{$Flush, PartitionIndex, DataFile_{N}$}
   \State $Buffer \gets $ \Call{new}{}
 \EndIf
\EndFor
\State \Call{Flush}{$Buffer, PartitionIndex, DataFile_{N}$}
\EndProcedure
\Statex
\Procedure{Flush}{$Buffer, PartitionIndex, DataFile_{N}$}
   \State $PartitionIndex \Leftarrow \langle $\Call{Token}{$PartitionKey$}$ \mapsto DataFile_{0}.Pos \rangle \mid \forall PartitionKey \in Buffer$
   \State $DataFile_{0} \Leftarrow \lvert Buffer \rvert$
   \State $DataFile_{0} \Leftarrow $ \Call{Token}{$PartitionKey$} $ \mid \forall \langle PartitionKey, \_,\_,\_\rangle \in Buffer \rangle$
   \State $Offset \gets 0$
   \For{$\langle PartitionKey, \_, \_, \_ \rangle \gets Buffer$}
     \State $DataFile_0 \Leftarrow Offset$
     \State $Offset \gets Offset + $ \Call{SizeOf}{$\langle PartitionKey, RowIndex, ValueStore_0, $ \textsc{Epoch}$,$ \{\textsc{Address}$^N$\}$\rangle$}
   \EndFor
   \For{$\langle PartitionKey, PartitionEpoch, RowIndex, ValueStore_{N} \rangle \gets Buffer$}
    \State $DataFile_{0} \Leftarrow \langle PartitionEpoch, PartitionKey \rangle$
    \State $Pos_{n} \gets DataFile_{n}.Pos \mid \forall n \in N$
    \State $Pos_{0} \gets Pos_0 + $ \Call{SizeOf}{$RowIndex$} $ + $ \Call{SizeOf}{\{\textsc{Address$^N$}\}}
    \State $DataFile_{0} \Leftarrow Pos_{N}$
    \State $DataFile_{0} \Leftarrow RowIndex$
    \State $DataFile_{n} \Leftarrow ValueStore_{n} \mid \forall n \in N$
    \State Pad $DataFile_{n}$ to a page boundary $\mid \forall n \in N$
   \EndFor
\EndProcedure
\end{algorithmic}
\end{algorithm}

\begin{algorithm}
\scriptsize
\caption{Reading}
\begin{algorithmic}[2]
\Function{Read}{$PartitionKey, Cols, RowKeys$}
\State $ColGroup, ColId_N \dashleftarrow Metadata$
\State $ReadGroup_N \gets $ \Call{ReadGroups}{$Cols, GolGroup_N$}
\State $Token \gets $ \Call{Token}{$PartitionKey$}
\State $CandidatePages \gets $\Call{Lookup}{$PartitionIndex, Token$}
\For{$Page \gets CandidatePages$}
 \State $Count \dashleftarrow DataFile_{0}[Page]$
 \State $TokenBase \gets Page +  $ \Call{SizeOf}{\textsc{$Count$}}
 \State $OffsetsBase \gets TokenBase + Count \times $\Call{SizeOf}{\textsc{Token}}
 \For{$i \gets $ \Call{Find}{$DataFile_0[TokenBase \dots OffsetBase), Token$}}
  \State $OffsetPosition \gets OffsetsBase + i \times $ \Call{SizeOf}{\textsc{Offset}}
  \State $Offset \dashleftarrow DataFile_{0}[OffsetPosition]$
  \State $PartitionEpoch, CandidateKey, Pos_N \dashleftarrow DataFile_{0}[Offset]$
  \If{$PartitionKey = CandidateKey$} 
   \State $Offset \gets Offset + $ \Call{SizeOf}{$\langle PartitionEpoch, CandidateKey, Pos_N \rangle$}
   \State $RowIndex \gets DataFile_0[Offset]$
   \State \Return \Call{Build}{$RowKeys, PartitionEpoch, ReadGroup_N, ColId_N, RowIndex, Pos_N, DataFile_N$}
  \EndIf
 \EndFor
\EndFor
\Return \textbf{nil}
\EndFunction
\Statex
\Function{Build}{$RowKeys, PartitionEpoch, ReadGroup_N, ColId_N, Pos_N, DataFile_N$}
\State $Partition \gets $ \Call{New}{}
\State $ValueStore_n \Leftarrow \langle DataFile_n[Pos_n] \rangle \mid \forall n \in N$
\For{$RowKey \gets RowKeys$}
 \State $Row \gets $ \Call{New}{}
 \State $Pos_N, EpochDelta \gets$ \Call{Read}{$RowIndex, RowKey$}
 \State $RowEpoch \gets PartitionEpoch + EpochDelta$
 \For{$n \gets N : ReadGroup_n \neq \emptyset$}
  \State $Row \Leftarrow $ \Call{Read}{$ValueStore, ColId_n(ReadGroup_n), Pos_n, RowEpoch$}
 \EndFor
 \State $Partition \Leftarrow Row$
\EndFor
\Return $Partition$
\EndFunction
\end{algorithmic}
\end{algorithm}

\clearpage
\subsection{Extension Two - Partition Reordering}
\small
    An sstable represents an ordered collection of partitions, however this doesn't
    actually require the data be in order on disk. If we permit the records to appear
    out of order, this could permit us to ensure few records cross a page boundary,
    reducing the number of IOPs needed per query, increasing system throughput and reducing
    maximum query latency. For linear scan performance we want records to be \emph{almost} 
    in order, so we define a $BufferLimit$ which bounds the amount of buffer space any reader can
    be required to use to ensure it can produce its results in their true order, and at flush we 
    ensure this property is enforced. We do this by buffering some multiple $K$ of $BufferLimit$ during 
    flush, and selecting from this buffer a well packed page to write that favours records earlier
    on in the stream, while leaving a good distribution of records to help pack future pages.

\begin{algorithm}
\scriptsize
\caption{Writing}
\begin{algorithmic}[1]
\Procedure{Write}{$Partitions, Group_N$}
\State $\dots$
\Statex
\For{$p \gets Partitions$} 
 \State $\dots$
 \State $Distance \gets \sum\limits_{n=0}^{N} (DataFile_{n}.pos - \min\limits_{\forall b \in Buffer} b.Pos_{n}) $
 \If {$Distance \geq K \times BufferLimit$}
   \State $Flush \gets$ \Call{Select}{$Buffer$}
   \State \Call{Flush}{$Flush, PartitionIndex, DataFile_{N}$}
   \State $Buffer \gets Buffer \setminus Flush$
 \EndIf
\EndFor
\EndProcedure
\Statex
\Function{Select}{$Buffer$}
\State $Selection \gets$ ?
\Ensure $Selection \subseteq Buffer$
\Ensure $s \in Selection \cdot s.Pos = \min\limits_{\forall b \in Buffer} b.Pos_{n}$
\Ensure ($\exists k \cdot \sum\limits_{s \in Selection} s.Size_{n} \approx k \times PageSize) \mid \forall n \in N $
\Ensure Future calls to \Call{Select}{} can also meet criteria
\State \Return $Selection$
\EndFunction
\end{algorithmic}
\end{algorithm}

\begin{algorithm}
\scriptsize
\caption{Reading}
\begin{algorithmic}[2]
\Procedure{Scan}{$StartToken, EndToken, Cols, RowKeys, Out$}
\State $ColGroup, ColId_N \dashleftarrow Metadata$
\State $Read_N \gets $ \Call{ReadGroups}{$Cols, GolGroup_N$}
\For{$Page \gets $\Call{Range}{$PartitionIndex, Start, End$}}
 \State $Count \dashleftarrow DataFile_{0}[Page]$
 \State $TokenBase \gets Page +  $ \Call{SizeOf}{\textsc{$Count$}}
 \State $OffsetsBase \gets TokenBase + Count \times $\Call{SizeOf}{\textsc{Token}}
 \For{$0 \leq i < Count$}
  \State $TokenPosition \gets BasePos + i \times $ \Call{SizeOf}{\textsc{Token}}
  \State $Token \gets DataFile_0[TokenPosition]$
  \If{$StartToken \leq Token < EndToken$}
   \State $OffsetPosition \gets OffsetsBase + i \times $ \Call{SizeOf}{\textsc{Offset}}
   \State $Offset \dashleftarrow DataFile_{0}[OffsetPosition]$
   \State $PartitionEpoch, PartitionKey, Pos_N, RowIndex \dashleftarrow DataFile_{0}[Offset]$
   \State $Buffer \Leftarrow $ \Call{Build}{$RowKeys, PartitionEpoch, ReadGroup_N, ColId_N, RowIndex, Pos_N, DataFile_N$}
   \State $BufferSize \gets $ \Call{SizeOf}{$Buffer$}
   \If{$BufferSize \geq BufferLimit$}
    \State $Out \Leftarrow \min(Buffer)$
    \State $Buffer \gets Buffer / {\min(Buffer)}$
   \EndIf
  \EndIf
 \EndFor
\EndFor
\EndProcedure
\end{algorithmic}
\end{algorithm}

\clearpage
\subsection{Notes}
\small
\paragraph{Large Partitions}
\paragraph{}
    These pseudocode implementations have ignored certain complexities of dealing with very large
    partitions, and assumes we can assemble the entire partition in memory. This was to keep
    the complexity to a minimum. To solve this problem,
    partitions with a $RowIndex$ spread across many pages can be filtered into a separate
    file, with $Group_0$ being shifted to $Group_{-1}$, so that no values are stored alongside
    the $RowIndex$. Thus we do not need to know its length, since there is no $ValueStore_0$ to locate.
    Since only one partition will be serialized we also do not need to know the position of the next $RowIndex$.
\paragraph{Dynamic Block Size}
\paragraph{}
    At a minimum each file should select a block size optimal for the data distribution being written to it, 
    and the storage medium. For SSDs this should mean a block size just large enough to fit a single partition, 
    or just its row index, if such a block size exists; otherwise, a 4Kb block size (also a minimum block size).
    For spinning disks, a larger block size can be used unconditionally; say 32-64Kb, as is currently the case.
    We should consider/explore using a dynamic block size within each file.
\paragraph{Decoupling of $ValueStore_N$ from $DataFile_N$}
\paragraph{}
    There's not actually a requirement that each $ValueStore_n$ has its own $DataFile_n$, nor that a collection
    of alike $ValueStore_N$ could not be grouped to be indexed by the same $RowIndex$ entry (pure column oriented
    value stores can all be indexed by the same key). This is a simple addition, but complicates the pseudocode
    unnecessarily.

\clearpage
\section{Component Variants}
Here we will outline some categories and details of implementation variants for each component.

\subsection{Partition Index}
The partition index most likely needs just two implementations, although it may be that subtle variations
of each are tried over time. The implementations depend on the nature of partitioning scheme, which
fall into two categories: Ordered and Hashed.
\paragraph{Ordered}
\paragraph{}
    The obvious choice is a B\textsuperscript{+}-tree, although since our data is static some alternatives are viable.
    Since this partitioning scheme is uncommon, however, the labour may not be worth investing in a new approach,
    and a simplification of the current scheme may be sensible. The only difference is that each page has its own
    mini-index, so we only need to store the start and end tokens for each page. This translates roughly to 
    the sstable representing the bottom three levels of a B\textsuperscript{+}-tree, with the highest 
    remaining resident in memory as our summary, and the middle being the highly cacheable index.
\paragraph{Hashed}
\paragraph{}
    There are a multitude of ways the hash layout can be exploited, that are out of scope for this document. 
    Here we will outline
    a method that in many cases should permit single IOPs for each read while occupying very little 
    resident memory, however the efficiency of the technique will vary based on partition size distributions.
    \\\\
    The principle is quite simple: given a good quality hash function (and in conjunction with 
    \href{https://issues.apache.org/jira/browse/CASSANDRA-6696}{CASSANDRA-6696}), if the partitions are
    all of equal size we can know exactly where to seek for a record that \textit{does} exist: if there
    are $N$ pages, representing $StartToken$ and $EndToken$, then we should look in page $N \times \dfrac{Token - StartToken}{EndToken - StartToken}$
    \\\\
    However, not all partitions will be the same size. So we perform this calculation for each stored token
    and compare against the truth. We build a tree that maps ranges of tokens to the maximum degree 
    of inaccuracy for that range, along with corrections for subranges that are actually out. The tree is
    then traversed until the inaccuracy is low enough to guarantee we are within the right page if the item
    exists in the table. For lookups touching the middle item in a page this will occur rapidly, whereas those
    on the edges may take a few more steps. For uniform data distributions and small records this should 
    permit strong compression. In cases where it does not, we can fallback to a simple array of the boundaries
    for each page, which can be indexed into in a similar fashion, with a binary search being performed within
    the probabilistic bounds the boundary should be found in.
\begin{algorithm}
\scriptsize
\caption{Hash Partition Lookup}
\begin{algorithmic}[2]
\Function{Lookup}{$PartitionIndex, PartitionKey$}
\State $Hash \gets $ \Call{Hash}{$PartitionKey$}
\State $Position$
\State \Return All positions possibly containing $PartitionKey$
\EndFunction

\end{algorithmic}
\end{algorithm}
    
\small
\paragraph{Out of Order Records}
\paragraph{}
    In both cases out-of-order persistence is permitted, and to support this a generalized pre-index
    may be useful, in which we store the few tokens we have permitted to be re-ordered. Lookups hit
    this index first, and on finding no results fallback to the normal index.

\subsection{Row Index}
The Row Index has by far the most scope for variation, so we will only touch briefly on the various
categories here, with a high level overview of their implications. We will mention retrieval and
merge performance, by intuitive description of characteristics only.
\paragraph{Entry Per Cell}
\paragraph{}
    This isn't really possible in the new world, but is worth discussing for comparison since it
    describes the current situation. Here we have each clustering prefix repeated once per cell.
    The cost of merging is as suboptimal as possible: 
    \begin{enumerate}
      \item There are $\lvert Columns \rvert$ more items to merge than necessary
      \item Shared clustering prefixes must be compared in full, so each comparison is costlier
      \item All rows must be compared; there is no pruning of known disjoint descendant sets 
    \end{enumerate}
\paragraph{Linear Collection}
\paragraph{}
    This is the closest to the current scenario, except that we only repeat the data once per row,
    and not once per cell. The idea would be to print each complete clustering prefix in sequence,
    and linearly scan for the relevant record. For very large partitions this would not support 
    any true indexing, but it is very simple, and for small partitions would be fine, and optimal
    for single rows. The cost of merging is improved by eliminating (1)
\paragraph{Column Trie, Linear Internal Collections}
\paragraph{}
    The next simplest approach is to split each clustering column into its own set of linear
    sequences of data, so that with multiple clustering columns we do not repeat data present
    in prior tiers. This permits data compression. Merging is also improved by significantly
    reducing the effects of (2) and (3). For partitions whose clustering column tiers are 
    each smaller than a single page this is a fairly optimal approach as binary search 
    can be performed on each tier as you descend.
\paragraph{Column Trie, BTree Internal Collections}
\paragraph{}
    When the trie levels are larger than a page, we need to introduce paging, and to do this a
    BTree makes perfect sense. This is still a very general collection, since any kind of comparison
    can be performed on each BTree item, so it can support all current and custom data types.
    This optimisation permits further improvement to (3) for merging, as well as optimal
    search costs for custom data types. 
\paragraph{Binary Trie}
\paragraph{}
    If the clustering prefixes can be compared in bit-order, a binary trie can be used.
    This permits optimal behaviour for all of (1), (2) and (3), making merges extremely
    cheap as the number of rows and partitions grow, which is likely to be of significant benefit,
    given how CPU constrained some users are on these costs. This also permits superior
    data compression. There are a number of possible binary trie variants to explore, but this
    warrants a separate document or JIRA ticket to discuss the options.

\subsection{Value Store}
The value store generally has only two major variants, and a hybrid between the two: column- and row-oriented.
There is no requirement that a given table subscribe to one or the other, however. Within a single sstable
there can be a mix, with each field grouping selecting its own value persistence approach.
\paragraph{Row Oriented}
\paragraph{}
This most closely resembles the current storage, except that we consider each row a discrete unit
and encode certain information prefixing it to permit indexed access within the row, and to permit
compression of the structural data.
\begin{itemize}
\scriptsize
  \item $HasCell$ Bitmap\\[2pt]
    Each column that occurs at least once in the file will have an index in this bitmap, which encodes
    if the column appears in this row, to permit efficient indexing within the row. It may be that some
    columns appear in every row, and these may be encoded in the file metadata to remove them from the
    row bitmaps
  \item $HasTimestamp$ Bitmap\\[2pt]
    Each column with its bit \textit{set} in $HasCell$ will have an index in $HasTimestamp$; this 
    will indicate if there is any timestamp offset necessary from $RowEpoch$; a value of zero 
    indicates $RowEpoch$ is enough by itself to construct the cell timestamp.
  \item $HasValue$ Bitmap\\[2pt]
    Each column with its bit \textit{set} in $HasCell$ will have an index in $HasValue$; this 
    will indicate if there is any actual data associated, or if the "value" is a tombstone
  \item $HasExpiry$ Bitmap\\[2pt]
    Each column with its bit \textit{set} in $HasValue$ will have an index in $HasExpiry$; this 
    will indicate if there is an expiry associated with the value.
  \item $HasInfo$ Bitmap\\[2pt]
    A bitmap indicating if $HasCell$, $HasTimestamp$, $HasValue$ or $HasExpiry$ are necessary to 
    encode.
  \item $TimestampLength$\\[2pt]
    If any bits are set in $HasTimestamp$, this will encode how many bytes are needed to encode them.
    Each will be encoded with the same width. This and $HasInfo$ can be encoded in the same byte.
  \item $Timestamps$\\[2pt]
    A fixed-width array of timestamps for each column marked in $HasTimestamp$
  \item $Expiries$\\[2pt]
    A fixed-width array of expiries for each column marked in $HasExpiry$
  \item $ColumnWidth : ColId \mapsto \mathbb{N}$\\[2pt]
    A table level property indicating which columns are fixed width, and their widths. A value of $\infty$
    indicates the column is dynamic width.
  \item $ColumnCount$\\[2pt]
    A table level property indicating the number of columns persisted against this value store
  \item $Values_B$\\[2pt]
    The cells appearing in $HasValue$ whose types permit fixed-width encoding of length $B$ will appear next, 
    so that they may indexed directly without any further information. All columns will appear grouped by
    size, but also in $ColId$ order; we will construct $ColId$ to enforce this at write-time.
  \item $Offsets$\\[2pt]
    The index of any dynamic length fields present in $HasValue$ will follow, so that they may also be 
    accessed directly. Encoded as $Values_{\infty}$.
  \item $DValues$\\[2pt]
    Finally we encode the dynamic length fields themselves.
\end{itemize}

\subparagraph{Notes}
\begin{itemize}
 \item{} It is necessary that our bitmaps support efficient rank operations, i.e. count the number of bits
 less than an index. For small bitmaps ($\leq$ 64 bits) this is trivial, but for larger bitmaps
 it requires a little extra data to implement efficiently. However it may be easiest to encode rows
 as linked-lists of $\leq$ 64 possible columns, since more should be rare.
 \item{} The pseudocode implementation assumes we read all of our bitmaps out of the row, but this is unnecessary
 and done only for clarity 
 \item{} To help read implementation, we select $ColId$ values that correspond to the size of the column, with 
 dynamically sized columns occurring last. This permits us to walk the columns that are being queried in the order
 the data is stored on disk, and to perform fewer calculations to know index into it, 
\end{itemize}

\begin{algorithm}
\scriptsize
\caption{Row Oriented Value Retrieval}
\begin{algorithmic}[2]
\Function{Read}{$ValueStore, ColIds, Pos, RowEpoch$}
\State $HasInfo, TimestampLength \dashleftarrow ValueStore$
\State $HasCell, HasValue \dashleftarrow ValueStore$
\State $HasTimestamp, HasExpiry, Timestamps, Expiries \gets \emptyset$
\If {$'HasCell' \in HasInfo$}
 \State $HasCell \dashleftarrow ValueStore[\dots $\tiny$\dfrac{ColumnCount}{8})$\scriptsize
\EndIf
\If {$'HasTimestamp' \in HasInfo$}
 \State $HasTimestamp \dashleftarrow ValueStore[\dots $\tiny$\dfrac{\lvert HasCell \rvert}{8})$\scriptsize
\EndIf
\If {$'HasValue' \in HasInfo$}
 \State $HasValue \dashleftarrow ValueStore[\dots $\tiny$\dfrac{\lvert HasCell \rvert}{8})$\scriptsize
\EndIf
\If {$'HasExpiry' \in HasInfo$}
 \State $HasExpiry\dashleftarrow ValueStore[\dots $\tiny$\dfrac{\lvert HasValue \rvert}{8})$\scriptsize
\EndIf
\If {$\lvert HasTimestamp \rvert > 0$}
 \State $Timestamps \dashleftarrow ValueStore[\dots $\tiny$\lvert HasTimestamp \rvert \times TimestampLength)$\scriptsize
\EndIf
\If {$\lvert HasExpiry \rvert > 0$}
 \State $Expiries \dashleftarrow ValueStore[\dots $\tiny$\lvert HasExpiry \rvert \times 8)$\scriptsize
\EndIf
\Statex
\State $Row \gets $ \Call{New}{}

\State $Width, WidthStart, WidthEnd \gets 0, 0, 0$
\For{$ColId \gets ColIds$}
 \If{$ColId \in HasValue$}
  \State $Expiry \gets 0$
  \State $Timestamp \gets RowEpoch$
  \If{$ColId \in HasTimestamp$}
   \State $Timestamp \gets Timestamp + Timestamps[$\Call{Rank}{$ColId, HasTimestamp$}$]$
  \EndIf
  \If{$ColId \notin HasValue$}
   \State $Row \Leftarrow \langle ColId, Timestamp, 'Deleted', Expiry \rangle$
  \Else
   \If{$ColId \in HasExpiry$}
    \State $Expiry \gets Expiries[$\Call{Rank}{$ColId, HasExpiry$}$]$
   \EndIf
   \If{$ColumnWidth(ColId) \neq Width$}
    \State $PrevWidth, Width \gets Width, ColumnWidth(ColId)$
    \For{$Fetch \gets \{ w \mid w \in \mathbf{ran}(Width), PrevWidth < w \leq Width\}$}
     \State $LastColId \gets \max{\{ColId \mid ColumnWidth(ColId) = Fetch\}}$
     \State $WithStart \gets WidthEnd$
     \State $WidthEnd \gets 1 + $ \Call{Rank}{$HasValue,LastColId$}
     \State $Count \gets WidthEnd - WidthStart$
     \State $Values_{Fetch} \dashleftarrow ValueStore[\dots Count \times Fetch)$
    \EndFor
   \EndIf
   \If{$Width \neq \infty$}
    \State $Value \dashleftarrow Values_{Width}[WidthStart + $ \Call{Rank}{$HasValue,ColId$}$]$
   \Else
    \State $Index \gets WidthStart + $ \Call{Rank}{$HasValue,ColId$}
    \State $StartOffset \dashleftarrow Values_{\infty}[Index]$
    \State $EndOffset \dashleftarrow Values_{\infty}[Index + 1]$
    \State $Value \dashleftarrow ValueStore[StartOffset \dots EndOffset)$
   \EndIf
   \State $Row \Leftarrow \langle ColId, Timestamp, Value, Expiry \rangle$
  \EndIf
 \EndIf
\EndFor
\State \Return $Row$
\EndFunction
\end{algorithmic}
\end{algorithm}

\clearpage
\paragraph{Column Oriented}
\paragraph{}
Full column-oriented storage supports only fixed-width data types, and is comparatively trivial to define:
each row is assigned an index within the partition, and every single column-oriented store multiplies this
with its base offset and the width of the data type to locate the position where data is stored. No muss, no fuss.
\paragraph{Hybrid}\normalsize{(Fixed-Width Block Encoding)}
\paragraph{}
The idea here is to exploit fixed-width and densely populated data distributions to further compress
data storage requirements, and make indexing to a given row/column involve fewer computational steps.
The basic idea is to store all of the values in a column-oriented fashion within a single data page only,
so a row index would store the page number, and the row offset within the page. The goal is to support
row-oriented workloads, just more efficiently. This model could support non-fixed-width types, but there
is probably little benefit to be had when these could be stored in a row-oriented scheme as easily.
 \begin{itemize}
\scriptsize
  \item $TimestampSize$\\[2pt]
    Occupying 2 bits per field in the row, indicates the size of timestamp encoding, with 0 indicating
    all values are equal to the row epoch; 1: 2 bytes; 2: 4 bytes; 3: 8 bytes.
    This must support efficient rank operations, like bitmaps for row-oriented storage, but based on
    cumulative value as opposed to index.
  \item $IsDeleted$ Bitmap[2pt]
    Encoded once per row, if $HasDeleted$ is set. A bitmap to be checked before returning the value
    encoded, that indicates is the cell is deleted. Treated as a pseudo column, of fixed width
  \item $HasDeleted$ Boolean\\[2pt]
    Indicates if $IsDeleted$ is present
  \item $HasExpiry$ Bitmap\\[2pt]
    Bitmap indicating which columns need to save expiry and deleted information
  \item $ColumnWidth$ Bitmap\\[2pt]
    Must also support a cumulative value rank based on fixed-width size, but this can be precomputed
  \item $ColumnCount$\\[2pt]
    A table level property indicating the number of columns persisted against this value store
\end{itemize}

\begin{algorithm}
\scriptsize
\caption{Hybrid Value Retrieval}
\begin{algorithmic}[2]
\Function{Read}{$ValueStore, ColIds, Pos, RowEpoch$}
\State $TimestampSize, HasExpiry, HasDeleted \dashleftarrow ValueStore$
\State $Row \gets $ \Call{New}{}
\State $IsDeletedId \gets ColumnCount$
\If{$HasDeleted$}
 \State $ColumnWidths \gets ColumnWidths \cup \{ IsDeletedId \mapsto $\tiny$\dfrac{IsDeletedId}{8}$\scriptsize$\}$
 \State $TimestampSize \gets TimestampSize \cup \{ IsDeletedId \mapsto 0\}$
 \State $ColIds \gets ColIds \cup \{ IsDeletedId \}$
\Else
 \State $ColumnWidths \gets ColumnWidths \cup \{ IsDeletedId \mapsto 0\}$
\EndIf

\State $RowLength \gets ColumnWidths(IsDeletedId) + \lvert HasExpiry \rvert \times 4 + \sum{TimestampSize}$
\State $IsDeleted \gets \emptyset$
\If{$HasDeleted$}
 \State $Index \gets RowLength \times (Pos + 1) - ColumnWidth(IsDeletedId)$
 \State $IsDeleted \dashleftarrow ValueStore[Index \dots Index + ColumnWidth(IsDeletedId))$
\EndIf
\For{$ColId \gets ColIds$}
 \State $Index \gets $\Call{Rank}{$HasExpiry, ColId$} $\times 4$
 \State $Index \gets Index + $\Call{Rank}{$HasTimestamp, ColId$}
 \State $Index \gets Index + $\Call{Rank}{$HasDeleted, ColId$}
 \State $Index \gets Index + $\Call{Rank}{$ColumnWidth, ColId$}
 \State $Expiry, Timestamp \gets 0, RowEpoch$
 \If{$ColId \in HasExpiry$}
  \State $Expiry \gets ValueStore[Index\dots Index + 4)$
  \State $Index /gets Index + 4$
 \EndIf 
 \State $Timestamp \gets ValueStore[Index \dots Index+TimestampSize(ColId)]$
 \State $Index \gets Index+TimestampSize(ColId)$
 \If{$ColId \in IsDeleted$}
  \State $Value \gets 'Deleted'$
 \Else
  \State $Value \gets ValueStore[Index \dots Index + ColumnWidth(ColId))$
 \EndIf
 \State $Row \Leftarrow \langle ColId, Timestamp, Value, Expiry \rangle$
\EndFor
\State \Return $Row$
\EndFunction
\end{algorithmic}
\end{algorithm}


\end{document}
