\documentclass[fleqn]{article}

\usepackage{hyperref}
\usepackage[utf8]{inputenc}
\usepackage{listings}
\usepackage{color}
\usepackage{xcolor}
\usepackage{amssymb}
\usepackage{amsmath}
\usepackage{setspace}
\usepackage{multirow}
\usepackage[]{algorithm2e}
\usepackage{xpatch}

\makeatletter
\makeatother

\title{Cassandra Storage Format\\Proposal for 3.1\\Version 1}
\author{Benedict Elliott Smith}
\date{January 2015}

\begin{document}

\maketitle

\begin{abstract}
This document is a draft proposal for a generalised replacement for the sstable format used in Cassandra.
The goal is to deliver a framework building on top of a number of abstracted components, with which
any efficient sstable representation can be encapsulated. It is designed to directly align with modern CQL
data modelling, however it should also support thrift models at least as efficiently.
\end{abstract}

\small

\section{Outline}
This section will outline the components and some associated concepts that are common to
\\
\paragraph{Definitions}
\subparagraph{$PartitionEpoch$, $RowEpoch$}
\subparagraph{}
    Against each tier of our structure we will encode the minimum timestamp found in the next
    tier, so that each may be encoded more efficiently. The expectation is that this will permit 
    many cells to store either only a few bytes, a single bit, or no data at all, to represent their timestamp.
\subparagraph{$ColId$}
\subparagraph{}
    A translation from column to a unique integer id

\paragraph{Components}
\paragraph{}
These components will each have more than one implementation.
\subparagraph{$PartitionIndex$}
\subparagraph{}
    One per sstable.
    Performs a translation from $PartitionKey$ to a set of addresses in $DataFile$ that may contain
    data for the key. This translation is permitted to yield false positives, but should
    expect to produce either zero or one result on average, with no false negatives.
    Multiple keys may map to the same location in $DataFile$, which can be exploited by an implementation
    to produce a compact mapping. These keys will generally be adjacent, but some may be out-of-order
    due to page packing. Its persistence is managed separately.
    \\

\subparagraph{$RowIndex$}
\subparagraph{}
    One per partition. Accessed sequentially, located by a $PartitionIndex$ lookup, it performs 
    a translation from $RowKey$ to a $RowEpoch$ and a position to provide to each $ValueStore_N$
    to produce cell values. It is stored in $DataFile$.
    \\

\subparagraph{$ValueStore$}
\subparagraph{}
    One per partition. A sequential chunk of bytes that combines the position retrieved
    from the $RowIndex$ with an offset stored in $DataFile_0$, the queried columns, the $ColId$ translation
    and the $RowEpoch$ to return cell values. We assume a property $Pos$ can be queried prior to insertion
    of a new row that returns the value we will store in $RowIndex$ for future lookups. It is stored in $DataFile$.
    \\


\section{Implementation Details}
\subsection{Notation Key}
\paragraph{}
\begin{tabular}{l l}
$X \Leftarrow Y$ & Write/Add/Append $Y$ To $X$\\
$X \dashleftarrow Y$ & Read $X$ From $Y$\\
$X \gets Y$ & Set $X$ to $Y$\\
$\langle A,B \rangle$ & A tuple containing the values $A$ and $B$\\
$\{Item | Constraint\}$ & The set of all $Item$ produced by $Constraint$\\
$S_N$ & $\equiv \{S_n | n \in N\}$\\
$Stmt_x | \forall x \in X$ & Execute statement $S$ for each $x$ in $X$\\
$\lvert X \rvert$ & $\equiv \mathbf{card}(X)$, i.e. the number of elements in $X$\\
\end{tabular}

\subsection{Core Outline}
\small
This section will outline the core algorithm using the component definitions provided in the previous
section. A basic description of the algorithm follows.
\\\\
\paragraph{Description}

\begin{algorithm}[H]
\scriptsize
\caption{Writing}
\SetKwFunction{Write}{Write}%
\SetKwFunction{Flush}{Flush}%
\Fn{\Write{$Partitions$}}{
}
\end{algorithm}

\end{document}
